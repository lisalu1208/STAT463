% Options for packages loaded elsewhere
\PassOptionsToPackage{unicode}{hyperref}
\PassOptionsToPackage{hyphens}{url}
%
\documentclass[
]{article}
\usepackage{lmodern}
\usepackage{amssymb,amsmath}
\usepackage{ifxetex,ifluatex}
\ifnum 0\ifxetex 1\fi\ifluatex 1\fi=0 % if pdftex
  \usepackage[T1]{fontenc}
  \usepackage[utf8]{inputenc}
  \usepackage{textcomp} % provide euro and other symbols
\else % if luatex or xetex
  \usepackage{unicode-math}
  \defaultfontfeatures{Scale=MatchLowercase}
  \defaultfontfeatures[\rmfamily]{Ligatures=TeX,Scale=1}
\fi
% Use upquote if available, for straight quotes in verbatim environments
\IfFileExists{upquote.sty}{\usepackage{upquote}}{}
\IfFileExists{microtype.sty}{% use microtype if available
  \usepackage[]{microtype}
  \UseMicrotypeSet[protrusion]{basicmath} % disable protrusion for tt fonts
}{}
\makeatletter
\@ifundefined{KOMAClassName}{% if non-KOMA class
  \IfFileExists{parskip.sty}{%
    \usepackage{parskip}
  }{% else
    \setlength{\parindent}{0pt}
    \setlength{\parskip}{6pt plus 2pt minus 1pt}}
}{% if KOMA class
  \KOMAoptions{parskip=half}}
\makeatother
\usepackage{xcolor}
\IfFileExists{xurl.sty}{\usepackage{xurl}}{} % add URL line breaks if available
\IfFileExists{bookmark.sty}{\usepackage{bookmark}}{\usepackage{hyperref}}
\hypersetup{
  pdftitle={Assignment 2 STAT 315-463: Multivariable Statistical Methods and Applications},
  hidelinks,
  pdfcreator={LaTeX via pandoc}}
\urlstyle{same} % disable monospaced font for URLs
\usepackage[margin=1in]{geometry}
\usepackage{graphicx,grffile}
\makeatletter
\def\maxwidth{\ifdim\Gin@nat@width>\linewidth\linewidth\else\Gin@nat@width\fi}
\def\maxheight{\ifdim\Gin@nat@height>\textheight\textheight\else\Gin@nat@height\fi}
\makeatother
% Scale images if necessary, so that they will not overflow the page
% margins by default, and it is still possible to overwrite the defaults
% using explicit options in \includegraphics[width, height, ...]{}
\setkeys{Gin}{width=\maxwidth,height=\maxheight,keepaspectratio}
% Set default figure placement to htbp
\makeatletter
\def\fps@figure{htbp}
\makeatother
\setlength{\emergencystretch}{3em} % prevent overfull lines
\providecommand{\tightlist}{%
  \setlength{\itemsep}{0pt}\setlength{\parskip}{0pt}}
\setcounter{secnumdepth}{-\maxdimen} % remove section numbering
\usepackage{bm}
\usepackage{amsmath}

\title{Assignment 2 STAT 315-463: Multivariable Statistical Methods and
Applications}
\author{}
\date{\vspace{-2.5em}}

\begin{document}
\maketitle

\vspace{-8 mm}

\textbf{Due date: Friday 24 March 2023}\\
\vspace{-7 mm}

\begin{itemize}
\tightlist
\item
  Your assignment needs to show the R code you used, and your well
  discussed answers to the questions.
\item
  Submit your assignments on Learn. \vspace{- 2 mm}
\end{itemize}

\hypertarget{background}{%
\subsection{Background}\label{background}}

In the dataset, \texttt{USJudgeRatings.csv}, you are presented with
ratings of State Judges on the Superior Court on 12 variables provided
by 43 Lawyers in 1977.\vspace{-5 mm}

\begin{center}
\begin{tabular}{l|l||l|l||l|l} \hline
CONT &  Number of contacts of lawyer with judge &
INTG &  Judicial integrity &
DMNR &  Demeanour \\
DILG &  Diligence &
CFMG &  Case flow managing &
DECI &  Prompt decisions \\
PREP &  Preparation for trial &
FAMI &  Familiarity with law &
ORAL &  Sound oral rulings \\
WRIT &  Sound written rulings &
PHYS &  Physical ability &
RTEN &  Worthy of retention \\ \hline
\end{tabular}\end{center}

\hypertarget{principal-component-analysis-of-the-rating-data.}{%
\subsection{Principal Component Analysis of the Rating
Data.}\label{principal-component-analysis-of-the-rating-data.}}

Perform a PCA on the standardised ratings. Note, you will need to
standardise the ratings yourself. Then answer the following questions.

\begin{enumerate}
\def\labelenumi{\arabic{enumi}.}
\item
  How many principal components do you believe should be retained.
  Justify your answer by looking at the variation in the data explained
  by each component.
\item
  In your own words, describe what you believe the first principal
  component is measuring.
\item
  What do you think the second principal component represents?
\item
  You are told \emph{Judicial Integrity} and \emph{Demeanour} are
  particularly important traits, and should be given 5 times the weight
  of the other variables. Re-run the Principal Component Analysis such
  that Integrity and Demeanour is given 5 times the weight of all other
  variables. What impact does this have?
\end{enumerate}

\hypertarget{factor-analysis-for-the-rating-data}{%
\subsection{Factor Analysis for the Rating
Data}\label{factor-analysis-for-the-rating-data}}

Perform Factor Analysis on the standardised Ratings Data.

\begin{enumerate}
\def\labelenumi{\arabic{enumi}.}
\item
  What happens when you try to fit a 3 and a 4 factor solution with no
  rotation. Hint: For the three factor solution, you may need to add
  \texttt{control=list(nstart=100)} as an additional argument in the
  \texttt{factanal} function.
\item
  Which variables are grouped by the first two factors? (e.g.~threshold
  \(|\text{loading}| \geq 0.25\))
\item
  Compare the factor loadings for the first two factors to the first two
  principal components of the standardised data found in the previous
  section. Comment on any similarities and/or differences.
\item
  Comment on the observed variable specific variances (the
  uniquenesses). Do you believe all observed variables are explained by
  the factors discovered.
\item
  Re-fit the 3 factor solution with a varimax rotation. How does this
  change the interpretation of the factors? In this case, do you find
  the rotated or non-rotated solution easier to interpret. Explain why
  or why not?
\end{enumerate}

\end{document}
